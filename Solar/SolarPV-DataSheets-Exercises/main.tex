%Preamble
% Some of these are recommended by "9 LaTeX packages everyone should use" http://www.howtotex.com/packages/9-essential-latex-packages-everyone-should-use/, 
%others were added as I came across the need for them.
\RequirePackage[l2tabu, orthodox]{nag} % has to go here in the preamble. Checks for obsolete stuff.
\documentclass{article} % the type of document
\pagestyle{plain}
\usepackage[utf8]{inputenc}
\usepackage[table, usenames, dvipsnames]{xcolor}
\definecolor{lightgray}{gray}{0.9} % I used this for the table background colour
\usepackage{multirow} % needed if cells in a table need to span more than one row
\usepackage{tabu} % makes table columns all have the same width
\usepackage{verbatim}
\usepackage[english]{babel} % so that the document follows the conventions of a  particular language eg english.
\usepackage[documents]{ragged2e}
\usepackage{amsmath} % for maths including align/align*
\usepackage[a4paper]{geometry} % for adjusting page margins
\usepackage[rightcaption]{sidecap}
\usepackage[skip=true,justification=justified,singlelinecheck=false,font=small,format=plain,labelfont=bf,up,textfont=normal,up]{caption}
\usepackage{graphicx} % needed for inserting figures eg \includegraphics
\graphicspath{ {figures/} } %Name of the folder in the project in which all the images are kept.
\usepackage{microtype} % improves spacing and general look of the document.
\usepackage{siunitx} % helps with units and numbers
\DeclareSIUnit\pence{p}
\DeclareSIUnit\kilowatthour{kWh}
\DeclareSIUnit\kilowattpeak{kW_p}
\usepackage{booktabs} % a way of creating tables without vertical separators. I haven't used this yet, and I don't think it is essential.
\usepackage[
backend=biber,
style=authoryear,
sorting=ynt
]{biblatex} % needed for citations and references
\addbibresource{references.bib} % name of the references file
\usepackage{gensymb} % not sure. Makes some symbols work.
\usepackage{csquotes}
\usepackage{hyperref} % needed for hyperlinks, internal and external. Needs to go at the end of the preamble, except for a few things eg cleveref
\hypersetup{
    colorlinks=true,
    linkcolor=blue,
    filecolor=magenta,      
    urlcolor=cyan,
} % style of the hyperlinks
\urlstyle{same}
\usepackage{cleveref}
%\usepackage{exercise} % does all the things with exercises and solutions.
\usepackage{exsheets} % does all the things with exercises and solutions.
\usepackage{enumitem}

%\\\\\\\\\\\\\\\\\\\\\\\\\\\\\\\\\\\\\\\\\\\\\\\\\\\\\\\\\\\\\\\\\\\\\\\\\\\\\\\\\\\\\\\\\\\\\\\\\\\\\\\\\\\\\\\\\\\\\\\\\\\\\\\\\\\\\\
% Title stuff
\title{Questions on Solar PV data sheets}
\author{Michael Hunt}
\date{}


\begin{document}
\maketitle

%\section*{Data}

%Specific heat of water $c$ = \SI{4.18}{\kilo\joule\per\kg\per\celsius}\\
%Density of water $\rho$ = \SI{1000}{\kg\per\metre\cubed}\\
%For solar resource data for Camborne, use the data provided in the spreadsheet %"Mean monthly radiation"

\section*{Exercises}
\begin{question}\label{qu:ex1}
From the datasheets for the BP350J PV module, answer the following questions
    \begin{enumerate}[label=\alph*)]
    \item	What is the maximum power output at standard test conditions?
    \item	What is the Voltage at which $P_\text{max}$ is obtained?
    \item	What is the Fill Factor of the panel?
    \item	What is the surface area of the irradiated part of the panel (i.e. the aperture)?
    \item	What is the surface area of the whole panel including the frame?
    \item	At standard test conditions, what is the efficiency of the silicon material (for this use the aperture area)
    \item	At standard test conditions, what is the efficiency of the whole panel (for this use the whole area including the frame)
    \item	Using the temperature coefficient of power, estimate the $P_\text{max}$ value at
        \begin{itemize}
        \item	\SI{0}{\celsius}
        \item	\SI{45}{\celsius}
        \end{itemize}
    \item	If the panel is operated at \SI{12}{V} (i.e. to charge a battery), use the $IV$ curve to estimate its power output under standard test conditions. Why might a DC to DC converter be necessary in a charge controller?
    \end{enumerate}
\end{question}

\begin{solution}\label{sol:ex1}
    \begin{enumerate}[label=\alph*)]
    \item	$P_\text{max}$=\SI{50}{W}
    \item	$V$ at $P_\text{max}$= \SI{17.5}{V}
    \item	$FF=\frac{P_\text{max}}{V_\text{OC}\times I_\text{SC}}=\frac{50}{21.8\times 3.17}=0.72$
    \item $\text{length}=839-(2\times 11.1)=\SI{816.8}{\milli\metre}=\SI{0.8168}{\metre}$\\
    $\text{width}=537-(2\times 11.1)=\SI{514.8}{\milli\metre}=\SI{0.5148}{\metre}$\\
    $\text{area}=0.8168\times 0.5148=\SI{0.420}{\metre\squared}$
    \item $\text{area}=0.839\times 0.537=\SI{0.451}{\metre\squared}$
    \item $\eta=\frac{P_\text{out}}{P_\text{in}}=\frac{P_\text{out}}{\text{irradiance}}=\frac{50}{0.42\times 1000}=\SI{11.9}{\percent}$
    \item $\eta=\frac{P_\text{out}}{P_\text{in}}=\frac{P_\text{out}}{\text{irradiance}}=\frac{50}{0.45\times 1000}=\SI{11.1}{\percent}$
    \item $P_\text{max}$ decreases by \SI{0.5}{\percent} per \si{\celsius} increase in temperature. The standard test conditions are at \SI{25}{\celsius}.
        \begin{itemize}
        \item	This means that at \SI{0}{\celsius } $P_\text{max}$ \emph{increases} from the STC value by $25\times \frac{0.5}{100}\times 50=\SI{6.25}{W}$\\
        Hence, at \SI{0}{\celsius}, $P_\text{max}=50+6.25=\SI{56.25}{W}$
        \item	At \SI{45}{\celsius } $P_\text{max}$ \emph{decreases} from the STC value by $20\times \frac{0.5}{100}\times 50=\SI{5.0}{W}$\\
        Hence, at \SI{0}{\celsius}, $P_\text{max}=50-5=\SI{45}{W}$
        \end{itemize}
\item From the $IV$ curve, $I$ at \SI{12}{V}=\SI{3.2}{\ampere}\\
$P=VI=12\times 3.2=\SI{38.4}{\watt}$\\
This is about \SI{23}{\percent} down on $P_\text{max}$ !
        
\end{enumerate}

\end{solution}

\begin{question}\label{qu:ex2}
Repeat the calculations above for the BP4175T module. Are there any differences?
\end{question}
\begin{solution}\label{sol:ex2}
    \begin{enumerate}[label=\alph*)]
    \item	$P_\text{max}$=\SI{175}{W}
    \item	$V$ at $P_\text{max}$= \SI{35.4}{V}
    \item	$FF=\frac{P_\text{max}}{V_\text{OC}\times I_\text{SC}}=\frac{175}{43.6\times 5.45}=0.736$
    \item Assuming the frame width is \SI{11.1}{\milli\metre} as for the BP350,\\$\text{aperture length}=1587-(2\times 11.1)=\SI{1564.8}{\milli\metre}=\SI{1.5648}{\metre}$\\
    $\text{aperture width}=790-(2\times 11.1)=\SI{767.8}{\milli\metre}=\SI{0.7678}{\metre}$\\
    $\text{aperture area}=1.5648\times 0.7678=\SI{1.20}{\metre\squared}$
    \item $\text{total area}=1.587\times 0.790=\SI{1.25}{\metre\squared}$
    \item $\eta=\frac{P_\text{out}}{P_\text{in}}=\frac{P_\text{out}}{\text{irradiance}}=\frac{175}{1.2\times 1000}=\SI{14.6}{\percent}$
    \item $\eta=\frac{P_\text{out}}{P_\text{in}}=\frac{P_\text{out}}{\text{irradiance}}=\frac{175}{1.25\times 1000}=\SI{14.0}{\percent}$. This is the value quoted in the spec.
    \item $P_\text{max}$ decreases by \SI{0.5}{\percent} per \si{\celsius} increase in temperature. The standard test conditions are at \SI{25}{\celsius}.
        \begin{itemize}
        \item	This means that at \SI{0}{\celsius } $P_\text{max}$ \emph{increases} from the STC value by $25\times \frac{0.5}{100}\times 175=\SI{21.9}{W}$\\
        Hence, at \SI{0}{\celsius}, $P_\text{max}=175+21.9=\SI{196.9}{W}$
        \item	At \SI{45}{\celsius } $P_\text{max}$ \emph{decreases} from the STC value by $20\times \frac{0.5}{100}\times 175=\SI{17.5}{W}$\\
        Hence, at \SI{0}{\celsius}, $P_\text{max}=175-17.5=\SI{157.5}{W}$
        \end{itemize}
\item From the $IV$ curve, $I$ at \SI{12}{V}=\SI{5.5}{\ampere}\\
$P=VI=12\times 5.5=\SI{66}{\watt}$\\
Which is a lot less than \SI{175}{\watt}!
\end{enumerate}
\end{solution}

\begin{question}\label{qu:ex3}

A roof is south facing at slope \SI{35}{\degree}. It is \SI{8}{\metre} long and \SI{4}{\metre} wide. The annual irradiation at this site is \SI{1240}{\kilowatthour\per\metre\squared}\\Establish
\begin{enumerate}[label=\alph*)]
\item How many BP350J panels could be mounted on the roof, if there must be a 1m border all the way around to allow for wind loads and maintenance?
\item What peak power is this system?
\item	What area do these panels cover?
\item	Based on the efficiency you have calculated Exercise 1 (g), what electrical output could be expected from this system each year?
\item	If the output is reduced by \SI{24}{\percent} for temperature, reflection and cabling losses, what is the actual likely energy output from the system (figure used by PVGIS).
\item	A BP350J costs £217.38, how much do the panels in this array cost?
\item	Assuming that the system lasts 25 years, and excluding the other costs associated with installing such a system, how much would it cost per kWh of generated electricity? How does this compare with the cost of electricity?
\end{enumerate}
\end{question}
\begin{solution}
\begin{enumerate}[label=\alph*)]
\item $\text{Usable area}=2\times6=\SI{12}{\metre\squared}$. Can fit 2 modules $\times$ 11 along = 22 modules
\item $P_\text{max}=22\times 50=\SI{1.1}{\kilowattpeak}$
\item $\text{Area}=22\times (0.839\times0.537)=\SI{9.91}{\metre\squared}$
\item $P_\text{out}=\eta \times P_\text{in}$ and $E_\text{out}=\eta \times E_\text{in}=\eta \times \text{Area}\times\text{Irradation}$\\Hence\\$E_\text{out}=\frac{11.1}{100}\times 9.91\times 1240=\SI{1364}{\kilowatthour}$
\item \SI{24}{\percent} of 1364 = \SI{327}{\kilowatthour} This is the energy lost per year. hence the net annual output of the system is $E_\text{out}=1364-327=\SI{1037}{\kilowatthour\per\year}$. This means a net annual output of \SI{943}{\kilowatthour} per \si{\kilowattpeak}.
\item Total cost $=22\times 217.38=$£4782
\item Total energy production over 25 years $=25\times
1037=\SI{25925}{\kilowatthour}$\\
Cost per kWh over 25 years $=\frac{4782}{25925}=$£0.18 per kWh
\end{enumerate}
\end{solution}
\begin{question}
Repeat Exercise 3 but with the BP4175 module, which cost £495.65 per module.
\end{question}
\begin{solution}
\begin{enumerate}[label=\alph*)]
\item 1 row x 7 along = 7 modules
\item $P_\text{max}=7\times 175=\SI{1.225}{\kilowattpeak}$
\item $\text{Area}=7\times (1.587\times0.790)=\SI{8.78}{\metre\squared}$
\item $P_\text{out}=\eta \times P_\text{in}$ and $E_\text{out}=\eta \times E_\text{in}=\eta \times \text{Area}\times\text{Irradation}$\\Hence\\$E_\text{out}=\frac{14}{100}\times 8.78\times 1240=\SI{1510}{\kilowatthour}$
\item \SI{24}{\percent} of 1510 = \SI{362}{\kilowatthour} This is the energy lost per year. hence the net annual output of the system is $E_\text{out}=1510-362=\SI{1147}{\kilowatthour\per\year}$. This means a net annual output of \SI{937}{\kilowatthour} per \si{\kilowattpeak}.
\item Total cost $=7\times 495.65=$£3469.55
\item Total energy production over 25 years $=25\times
1147=\SI{28675}{\kilowatthour}$\\
Cost per kWh over 25 years $=\frac{3469.55}{28675}=$£0.12 per kWh
\end{enumerate}
You can see that the electricity produced remains expensive. A FiT or other subsidy is necessary to make the cost of production competitive.
\end{solution}
%\printbibliography
\pagebreak
\section*{Solutions}
\printsolutions
\end{document}
