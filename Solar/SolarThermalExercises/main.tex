%Preamble
% Some of these are recommended by "9 LaTeX packages everyone should use" http://www.howtotex.com/packages/9-essential-latex-packages-everyone-should-use/, 
%others were added as I came across the need for them.
\RequirePackage[l2tabu, orthodox]{nag} % has to go here in the preamble. Checks for obsolete stuff.
\documentclass{article} % the type of document
\pagestyle{plain}
\usepackage[utf8]{inputenc}
\usepackage[table, usenames, dvipsnames]{xcolor}
\definecolor{lightgray}{gray}{0.9} % I used this for the table background colour
\usepackage{multirow} % needed if cells in a table need to span more than one row
\usepackage{tabu} % makes table columns all have the same width
\usepackage{verbatim}
\usepackage[english]{babel} % so that the document follows the conventions of a  particular language eg english.
\usepackage[documents]{ragged2e}
\usepackage{amsmath} % for maths including align/align*
\usepackage[a4paper]{geometry} % for adjusting page margins
\usepackage[rightcaption]{sidecap}
\usepackage[skip=true,justification=justified,singlelinecheck=false,font=small,format=plain,labelfont=bf,up,textfont=normal,up]{caption}
\usepackage{graphicx} % needed for inserting figures eg \includegraphics
\graphicspath{ {figures/} } %Name of the folder in the project in which all the images are kept.
\usepackage{microtype} % improves spacing and general look of the document.
\usepackage{siunitx} % helps with units and numbers
\DeclareSIUnit\pence{p}
\DeclareSIUnit\kilowatthour{kWh}
\usepackage{booktabs} % a way of creating tables without vertical separators. I haven't used this yet, and I don't think it is essential.
\usepackage[
backend=biber,
style=authoryear,
sorting=ynt
]{biblatex} % needed for citations and references
\addbibresource{references.bib} % name of the references file
\usepackage{gensymb} % not sure. Makes some symbols work.
\usepackage{csquotes}
\usepackage{hyperref} % needed for hyperlinks, internal and external. Needs to go at the end of the preamble, except for a few things eg cleveref
\hypersetup{
    colorlinks=true,
    linkcolor=blue,
    filecolor=magenta,      
    urlcolor=cyan,
} % style of the hyperlinks
\urlstyle{same}
\usepackage{cleveref}
%\usepackage{exercise} % does all the things with exercises and solutions.
\usepackage{exsheets} % does all the things with exercises and solutions.
\usepackage{enumitem}

%\\\\\\\\\\\\\\\\\\\\\\\\\\\\\\\\\\\\\\\\\\\\\\\\\\\\\\\\\\\\\\\\\\\\\\\\\\\\\\\\\\\\\\\\\\\\\\\\\\\\\\\\\\\\\\\\\\\\\\\\\\\\\\\\\\\\\\
% Title stuff
\title{Questions on Solar thermal heating}
\author{Michael Hunt}
\date{}


\begin{document}
\maketitle

\section*{Data}

Specific heat of water $c$ = \SI{4.18}{\kilo\joule\per\kg\per\celsius}\\
Density of water $\rho$ = \SI{1000}{\kg\per\metre\cubed}\\
For solar resource data for Camborne, use the data provided in the spreadsheet "Mean monthly radiation"

\section*{Exercises}

\begin{question}\label{qu:ex1}
Estimate how much energy would be needed to heat the water in the following:
    \begin{enumerate}[label=\alph*)]
        \item A typical bath
        \item A washing machine which uses \SI{10}{\litre} every wash and rinse. Each cycle has 1 wash and 2 rinses at \SI{30}{\celsius} 
        \item A typical shower
        \item Washing up the dishes after a meal
    \end{enumerate}
\end{question} 
\begin{solution}\label{sol:ex1}
    \begin{enumerate}[label=\alph*)]
        \item A typical bath: assume inlet temperature of \SI{10}{\celsius} and a bath water temperature of \SI{40}{\celsius}.
            \begin{align*}
            \text{water\ volume} V&=0.15\times 0.4\times 1.4 = \SI{0.084}{\metre\cubed}\\
            \text{water\ mass}&=\rho V=1000\times 0.084=\SI{84}{\kg}\\
            \text{energy\ required } Q &=mc\Delta\theta\\
            &=0.084\times 4.18\times (40-10)\\
            &=\SI{10.6}{\mega\joule}\approx 3\ \text{kWh} \ (\SI{50}{\pence})
            \end{align*}
        \item A washing machine which uses \SI{10}{\litre} every wash and rinse. Each cycle has 1 wash and 2 rinses at \SI{30}{\celsius}
            \begin{align*}
            \text{mass}\ m &=\SI{30}{\kg}\\
            Q &=mc\Delta\theta\\
            &=30\times\ 4.18\ (30-10)\\
            &=\SI{2520}{\kilo\joule}\approx 0.7\ \text{kWh} \ (\SI{11}{\pence})
            \end{align*}
        \item A typical shower
            Assume flow rate of \SI{5}{\litre\per\minute} and that shower lasts 5 minutes, so 25 litre of water, which weighs 25 kg.
            \begin{align*}
            Q &=mc\Delta\theta\\
            &=25\times 4.18\times (40-10)\\
            &=\SI{3135}{\kilo\joule}\approx 0.9 \text{kWh} \ (\SI{14}{\pence})
            \end{align*}
        \item Washing up the dishes after a meal
            \text{Water temperature probably close to }\SI{50}{\celsius}
            \begin{align*}
            \text{water\ volume\ } V&=0.15\times 0.4\times 0.5 = \SI{0.03}{\metre\cubed}\\
            \text{water\ mass}&=\rho V=1000\times 0.03=\SI{30}{\kg}\\
            \text{energy\ required } Q &=mc\Delta\theta\\
            &=30\times 4.18\times (50-10)\\
            &=\SI{5016}{\kilo\joule}\approx 1.4\  \text{kWh} \ (\SI{22}{\pence})  
            \end{align*}
    \end{enumerate}
\end{solution}

\begin{question}\label{qu:ex2}
Estimate how much water could be heated in one day by \SI{1}{\metre\squared} of solar thermal panels on a roof in Camborne
    \begin{enumerate}[label=\alph*)]
        \item in the summer
        \item in the spring
        \item in winter
    \end{enumerate}
\end{question}
\begin{solution}
    \begin{enumerate}[label=\alph*)]
        \item Summer irradiation $\approx \SI{5.5}{\kilowatthour\per\metre\squared\per\day}$\\
        \SI{1}{\metre\squared} of collector would collect $5.5 \times \eta\ \si{\kilowatthour\per\day}$\
        \text{Assume\ }$\eta=0.6$\\
        \begin{align*}
        E&=5.5\eta\\
        &=\SI{3.3}{\kilowatthour\per\day}\approx \SI{11880}{\kilo\joule}\\
        Q&=mc\Delta\theta\\
        \end{align*}
        $\text{Assume\ } \Delta\theta=\SI{50}{\celsius}\text{\ from\ }\SI{10}{\celsius}\text{\ inlet,\ } \SI{60}{\celsius}\text{\ store temperature.}$\\
    \begin{align*}
    m&=\frac{Q}{c\Delta\theta}\\
   &=\frac{11880}{4.18\times 50}\\
    &=\SI{56.8}{\kg}\text{\ so\ about\ }\SI{57}{\litre}
    \end{align*}
        \item In spring daily irradiation $\approx  \SI{3}{\kilowatthour\per\metre\squared\per\day}$ so from previous answer, volume of water that can be heated $\approx\frac{3}{5.5}\times 57=\SI{31}{\litre}$
        \item in winter irradiation $< \SI{1}{\kilowatthour\per\metre\squared\per\day}$ so water volume that can be heated is at most $\approx\frac{1}{5.5}\times 57\approx\SI{10}{\litre}$
    \end{enumerate}
Actually, summer $\eta$ probably higher than 0.6 and winter $\eta$ probably less than 0.6, since irradiance is higher in summer and thermal losses would be lower. So, probably, summer heatable volume of water $> \SI{57}{\litre}$ and winter heatable volume $< \SI{10}{\litre}$
\end{solution}

\begin{question}\label{qu:ex3}
Based on your answers to question \ref{qu:ex2}, estimate the area of solar thermal panels that would be needed on a roof in Camborne to supply the hot water requirement of an average household:
    \begin{enumerate}[label=\alph*)]
        \item in the summer
        \item in the spring
        \item in winter
    \end{enumerate}
\end{question}
\begin{solution}
Assume the household hot water requirement $\approx \SI{150}{\litre\per\day}$. Thus the area of collector required to provide \emph{all} of this would be
    \begin{enumerate}[label=\alph*)]
        \item in the summer: $\approx \SI{3}{\metre\squared}$
        \item in the spring: $\approx \SI{5}{\metre\squared}$
        \item in winter: $\approx \SI{15}{\metre\squared}$
    \end{enumerate}
\end{solution}

\begin{question}
Show that a pump flow rate of 30-50 \si{\litre\per\hour} is reasonable in a solar thermal system in Camborne with \SI{4}{\metre\squared}, of collector area
\end{question}
\begin{solution}
If the maximum max solar intensity $\approx$ \SI{1000}{\watt\per\metre\squared}, collector area is \SI{4}{\metre\squared} and max.efficiency is $\eta \approx 0.7$ then max rate of energy capture is \SI{700}{\watt\metre\squared}, so about  \SI{2800}{\watt} in total. Mass of water that can be heated by \SI{50}{\celsius} per second will be
\begin{align*}
m&=\frac{Q}{c\Delta\theta}\\
&=\frac{2800}{4180\times 50}\\
&=\SI{0.013}{\kg\per\second}
\end{align*}
This is the required mass flow rate in the system. Since \SI{1}{\kg} of water has volume \SI{1}{\litre} the required volume flow rate is $\SI{0.013}{\litre\per\second}=3600\times 0.013= \SI{48}{\litre\per\hour}$
\end{solution}

\begin{question}
Show that for a swimming pool, a collector area of about 40\% of the pool area is reasonable.
\end{question}
\begin{solution}
First estimate heat $Q$ required per \si{\metre\squared} to heat pool up at start of season: if average depth =\SI{1.2}{\metre}, and $\Delta\theta = \SI{10}{\celsius}$ then
\begin{align*}
Q&=mc\Delta\theta\\
&=\rho Vc\Delta\theta\\
&=1000\times 1.2\times4.18\times10\\
&=\SI{50.16}{\mega\joule}\\&=\SI{13.9}{\kilowatthour}
\end{align*}
Suppose now that the the pool would cool down to ambient temperature in 10 days due to heat losses by conduction, convection and radiation. (This is a guess - we must do further work to refine it, but let's start here.) The rate of heat loss per square metre per day would therefore be $Q_\text{loss}=\frac{\SI{13.9}{\kilowatthour}}{10}=\SI{1.4}{\kilowatthour\per\day}$ If the irradiance during the summer months is \SI{5.5}{\kilowatthour\per\metre\squared} (true for southern England - subsititute another value for elsewhere) and if the efficiency of the solar collector system is $\eta=0.6$ then the energy collected per \si{\metre\squared} of collector would be
\begin{align*}
Q_\text{in}&=5.5\times\eta\\
&=5.5\times 0.6\\
&=\SI{3.3}{\kilowatthour\per\metre\squared}
\end{align*}
The area $A$ of collector required to make good thermal losses from each \SI{1}{\metre\squared} of the pool is thus
\begin{align*}
A&=\frac{1.39}{3.3}\\
&=\SI{0.42}{\metre\squared}
\end{align*}
Hence a collector area about 40\% of the pool area would be reasonable, \emph{if} the assumptions made hold good.
\end{solution}

%\printbibliography
\section*{Solutions}
\printsolutions
\end{document}
