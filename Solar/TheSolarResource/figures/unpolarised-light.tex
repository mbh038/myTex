\documentclass{standalone}
\usepackage{tikz}
\usetikzlibrary{arrows}



\renewcommand{\familydefault}{\sfdefault}

\begin{document}
\colorlet{crystal}{blue!75}

\def\zangle{-20}
\def\xangle{20}

\begin{tikzpicture}[x=(\xangle:0.75cm), y=(90:1cm), z=(\zangle:1.5cm),
    >=stealth, line cap=round, line join=round,
    lines/.style={gray!50, thick}, 
    axis/.style={black, thick},
    plate/.style={fill, opacity=0.875},
    markers/.style={orange, thick}]

\draw [axis, ->] (0,0,0) -- (0,0,6.25);

\begin{scope}[shift={(0,0,3.125)}]

%\node [yslant=tan(\zangle), above=0.25cm, align=center,font=\small] at 
%(1,1,1.5){Unpolarized Light};

\foreach \k [evaluate={%
    \i=\k*5.625; \j=\i>0 ? \i-5.625 : 0;
    \a=90-\i; 
    \b=90-\j; 
    \c=int((mod(\k,4)==0 && sin \a != 0) || (\k==65) || (\k==129)); 
    \d=int(\k+1/4);
    \r=(\k>64) ? 1.414 : 1;
    \xa=(\k > 64) && (\k < 129) ? 0 : sin(\a)*\r;
    \xb=(\k > 64) && (\k < 129) ? 0 : sin(\b)*\r;
    \ya=(\k < 129) ? sin(\a)*\r : 0;
    \yb=(\k < 129) ? sin(\b)*\r : 0;
    }] in {0,...,192}{
        \ifodd\d
            \ifnum\c=1
                \draw [ red,opacity=0.4,->] (0,0,\i/360) -- ++(\xa, \ya, 0);
            \fi
            \draw [red] (\xa, \ya, \i/360) -- (\xb, \yb, 
            \j/360);
        \else
            \draw [red] (\xa, \ya, \i/360) -- (\xb, \yb, 
            \j/360);
            \ifnum\c=1
                \draw [ red,opacity=0.4,->] (0,0,\i/360) -- ++(\xa, \ya, 0);
            \fi
        \fi
    }

\end{scope}

\begin{scope}%[shift={(0,0,3.125)}]

\node [yslant=tan(\zangle), above=0.25cm, align=center,font=\small] at 
(1,1,3){Unpolarized Light};

\foreach \k [evaluate={%
    \i=\k*5.625; \j=\i>0 ? \i-5.625 : 0;
    \a=90-\i; 
    \b=90-\j; 
    \c=int((mod(\k,4)==0 && sin \a != 0) || (\k==65) || (\k==129)); 
    \d=int(\k+1/4);
    \r=(\k>64) ? 1.414 : 1;
    \xa=(\k > 64) && (\k < 129) ? 0 : sin(\a)*\r;
    \xb=(\k > 64) && (\k < 129) ? 0 : sin(\b)*\r;
    \ya=(\k < 129) ? sin(\a)*\r : 0;
    \yb=(\k < 129) ? sin(\b)*\r : 0;
    }] in {0,...,192}{
        \ifodd\d
            \ifnum\c=1
                \draw [ red,opacity=0.4,->] (0,0,\i/360) -- ++(\xa, \ya, 0);
            \fi
            \draw [red] (\xa, \ya, \i/360) -- (\xb, \yb, 
            \j/360);
        \else
            \draw [red] (\xa, \ya, \i/360) -- (\xb, \yb, 
            \j/360);
            \ifnum\c=1
                \draw [ red,opacity=0.4,->] (0,0,\i/360) -- ++(\xa, \ya, 0);
            \fi
        \fi
    }

\end{scope}

\end{tikzpicture}

\end{document}
