\documentclass{standalone}
\usepackage{tikz}
\usetikzlibrary{arrows}



\renewcommand{\familydefault}{\sfdefault}

\begin{document}
\colorlet{crystal}{blue!75}

\def\zangle{-20}
\def\xangle{20}

\begin{tikzpicture}[x=(\xangle:0.75cm), y=(90:1cm), z=(\zangle:1.5cm),
    >=stealth, line cap=round, line join=round,
    lines/.style={gray!50, thick}, 
    axis/.style={black, thick},
    plate/.style={fill, opacity=0.875},
    markers/.style={orange, thick}]

\node [yslant=tan(\zangle), above=0.25cm, align=center,font=\small] at 
    (1,1,1.5){Left Handed \\ Circularly Polarized Light};

\draw [lines] (-1,-1,0) -- (-1,1,0) -- (1,1,0) -- (1,-1, 0) -- cycle;
\draw [lines] (1,0,0) \foreach \t in {0,5,...,355}{
        -- (cos \t, sin \t, 0) } -- cycle;

\draw [lines] (1,1,0) -- (1,1,3.125);
\draw [lines] (-1,-1,0) -- (-1,-1,3.125);
\draw [axis, ->] (0,0,3.125) -- (0,0,0);

\foreach \k [evaluate={%
    \i=\k*5.625; 
    \j=\i>0 ? \i-5.625 : 0; 
    \a=90-\i; 
    \b=90-\j; 
    \c=int(mod(\k,4));}] 
    in {0,...,192}{
        \ifnum\c=0
            \draw [->] (0,0,\i/360) -- ++(cos \a, sin \a, 0);
        \fi
        \draw [red] (cos \a, sin \a, \i/360) -- (cos \b, sin \b, \j/360);
    }

\begin{scope}[shift={(0,0,3.125)}]

\node [yslant=tan(\zangle), above=0.25cm, align=center,font=\small] at 
    (1,1,1.5){Linearly Polarized Light};

\begin{scope}[xscale=1.5, yscale=1.5]
    \path [crystal!25, plate] 
        (-1,-1,0) -- (-1,1,0) -- (1,1,0) -- (1,-1,0) -- cycle;
    \path [crystal!50, plate] 
        (-1,-1,0) -- (-1,-1,-0.125) -- (-1,1,-0.125) -- (-1,1, 0) -- cycle;
    \path [crystal!75, plate] 
        (-1,1,0) -- (-1,1,-0.125) -- (1,1,-0.125) -- (1,1, 0) -- cycle;
    \node [yslant=tan(\xangle), text=crystal!50, below, font=\small] at 
        (-1.125,-1,0){Quarter Wave Plate};
\end{scope}

\draw [markers] (0,1) -- (0,-1) (-0.5,0) -- (0.5,0);
\draw [lines] (1,1,0) -- (1,1,3);
\draw [lines] (-1,-1,0) -- (-1,-1,3);

\draw [axis] (0,0,0) -- (0,0,3);

\foreach \k [evaluate={%
    \i=\k*5.625; \j=\i>0 ? \i-5.625 : 0; 
    \a=90-\i; 
    \b=90-\j; 
    \c=int(mod(\k,4)==0 && sin \a != 0); 
    \d=int(\k+1/4);}] in {0,...,192}{
    \ifodd\d
        \ifnum\c=1
            \draw [->] (0,0,\i/360) -- ++(sin \a, sin \a, 0);
        \fi
        \draw [red] (sin \a, sin \a, \i/360) -- (sin \b, sin \b, \j/360);
    \else
        \draw [red] (sin \a, sin \a, \i/360) -- (sin \b, sin \b, \j/360);
        \ifnum\c=1
            \draw [->] (0,0,\i/360) -- ++(sin \a, sin \a, 0);
        \fi
    \fi
}
\end{scope}

\begin{scope}[shift={(0,0,6.125)}]

\node [yslant=tan(\zangle), above=0.25cm, align=center,font=\small] at 
(1,1,1.5){Unpolarized Light};

\begin{scope}[xscale=1.5, yscale=1.5]
    \path [crystal!25, plate] 
        (-1,-1,0) -- (-1,1,0) -- (1,1,0) -- (1,-1, 0) -- cycle;
    \path [crystal!50, plate] 
        (-1,-1,0) -- (-1,-1,-0.0625) -- (-1,1,-0.0625) -- (-1,1, 0) -- 
        cycle;
    \path [crystal!75, plate] 
        (-1,1,0) -- (-1,1,-0.0625) -- (1,1,-0.0625) -- (1,1, 0) -- cycle;
    \node [yslant=tan(\xangle), text=crystal!50, below, font=\small] at 
        (-1,-1,0){Linear Polarizer};
\end{scope}

\draw [markers] (-1.25,-1.25) -- (1.25,1.25);

\draw [lines] (0,1.414,0) -- (0,1.414,2);
\draw [lines] (1.414,0,0) -- (1.414,0,3);
\draw [lines] (1,1,0) -- (1,1,1);
\draw [lines] (-1,-1,0) -- (-1,-1, 0.5);
\draw [axis] (0,0,0) -- (0,0,3);

\foreach \k [evaluate={%
    \i=\k*5.625; \j=\i>0 ? \i-5.625 : 0;
    \a=90-\i; 
    \b=90-\j; 
    \c=int((mod(\k,4)==0 && sin \a != 0) || (\k==65) || (\k==129)); 
    \d=int(\k+1/4);
    \r=(\k>64) ? 1.414 : 1;
    \xa=(\k > 64) && (\k < 129) ? 0 : sin(\a)*\r;
    \xb=(\k > 64) && (\k < 129) ? 0 : sin(\b)*\r;
    \ya=(\k < 129) ? sin(\a)*\r : 0;
    \yb=(\k < 129) ? sin(\b)*\r : 0;
    }] in {0,...,192}{
        \ifodd\d
            \ifnum\c=1
                \draw [->] (0,0,\i/360) -- ++(\xa, \ya, 0);
            \fi
            \draw [red] (\xa, \ya, \i/360) -- (\xb, \yb, 
            \j/360);
        \else
            \draw [red] (\xa, \ya, \i/360) -- (\xb, \yb, 
            \j/360);
            \ifnum\c=1
                \draw [->] (0,0,\i/360) -- ++(\xa, \ya, 0);
            \fi
        \fi
    }

\draw [ultra thick, ->] (0,0,3.5) -- (0,0,3);

\end{scope}

\end{tikzpicture}

\end{document}
