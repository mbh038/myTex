\documentclass{standalone}
\usepackage{tikz}
\usetikzlibrary{arrows,shapes,positioning,calc,intersections,math}
\usetikzlibrary{decorations.markings}
\tikzstyle arrowstyle=[scale=1]
\tikzstyle directed=[postaction={decorate,decoration={markings,
    mark=at position .65 with {\arrow[arrowstyle]{stealth}}}}]
\tikzstyle reverse directed=[postaction={decorate,decoration={markings,
    mark=at position .65 with {\arrowreversed[arrowstyle]{stealth};}}}]

\begin{document}
\begin{tikzpicture}

\def\clipsize{4};
\def\incidentangle{60};
\def\rwater{1.333};
\def\rglass{1.5};

 \pgfmathsetmacro{\refractedangle}{asin(sin(\incidentangle)/\rglass)};
 \pgfmathsetmacro{\yincrement}{1/cos(\incidentangle)};
 \pgfmathsetmacro{\xincrement}{1/sin(\incidentangle)};
 \pgfmathsetmacro{\firstx}{-floor(\clipsize/tan(\incidentangle)+\clipsize)};
 \pgfmathsetmacro{\secondx}{\firstx+\xincrement};
 \pgfmathsetmacro{\lastx}{floor(\clipsize/tan(\refractedangle)+\clipsize)};
 \pgfmathsetmacro{\incidentlength}{\clipsize/sin(\incidentangle)};
  \pgfmathsetmacro{\refractedlength}{\clipsize/sin(\refractedangle)};
  
\clip (-\clipsize,-\clipsize) rectangle (\clipsize,\clipsize);  


    % define coordinates
    \coordinate (TL) at (-\clipsize, \clipsize);
    \coordinate (TR) at (\clipsize,\clipsize);
    \coordinate (BR) at (\clipsize,-\clipsize);
    \coordinate (BL) at (-\clipsize,-\clipsize);
    \coordinate (CL) at (-\clipsize,0);
    \coordinate (CR) at (\clipsize,0);
    \coordinate (CLL) at (-\clipsize,-0.01);
    \coordinate (CRL) at (\clipsize,-.01);
    \coordinate (CLH) at (-\clipsize,0.01);
    \coordinate (CRH) at (\clipsize,0.01);
    
    \coordinate (O) at (0,0) ;
    \coordinate (A) at (0,\clipsize) ;
    \coordinate (B) at (0,-\clipsize) ;
    

    % media
    \fill[blue!25!,opacity=.3] (-\clipsize,0) rectangle (\clipsize,\clipsize);
    \fill[blue!60!,opacity=.3] (-\clipsize,0) rectangle (\clipsize,-\clipsize);
    \node[right] at (-\clipsize+0.5,.25) {Air};
    \node[right] at (-\clipsize+0.5,-.25) {Glass};
    %\node[left] at (-2,-2) {Water};

    % axis
    \draw[dash pattern=on5pt off3pt] (A) -- (B);
    
    % rays
    \path [name path=air edge] (TL) -- (TR) -- (CRL) -- (CLL) -- cycle;
    \path [name path=glass edge] (CLH) -- (CRH) -- (BR) -- (BL) -- cycle;
    \path [name path=incident ray] (0,0)-- (90+\incidentangle:2*\clipsize);
    \path [name path=reflected ray] (0,0) -- (90-\incidentangle:2*\clipsize);
    \path [name path=refracted ray] (0,0)--(270+\refractedangle:2*\clipsize);  
    
    %incident ray
    \draw [name intersections={of=incident ray and air edge, by=t}] [red,ultra thick,reverse directed] (0,0)--
   node[near end,sloped,above]{incident ray} (t);
   %reflected ray
    \draw [name intersections={of=reflected ray and air edge, by=t}] [green!70,very thick,directed] (0,0)--
   node[green,near end,sloped,above]{reflected ray} (t);
    %refracted ray
     \draw [name intersections={of=refracted ray and glass edge, by=t}] [blue!80,very thick,directed] (0,0)--
  node[near end,sloped,above]{refracted ray} (t);
  
    % wavefronts
    \foreach \x in{\firstx,\secondx,...,\lastx}{
    \draw [red,thin] (\x,0) -- ++(\incidentangle:\incidentlength);
    \draw [green!70,very thin] (\x,0) -- ++(180-\incidentangle:\incidentlength);
    \draw [blue,very thin] (\x,0) -- ++(\refractedangle+180:\refractedlength);
    }
    
    % angles
    \draw [<->](0,1) arc (90:90+\incidentangle:1);
    \draw [<->](0,1) arc (90:90-\incidentangle:1);
    \draw [<->](0,-1.4) arc (270:270+\refractedangle:1.4);
    \node[] at (270+\refractedangle/2:1.8)  {$\theta_{2}$};
    \node[] at (90+\incidentangle/2:1.4)  {$\theta_{1}$};
    \node[] at (90-\incidentangle/2:1.4)  {$\theta_{1}$};
\end{tikzpicture}
\end{document}
