% Refraction
% Author: Jimi Oke
\documentclass{standalone}
\usepackage{tikz}
\usetikzlibrary{arrows,shapes,positioning,calc,intersections}
\usetikzlibrary{decorations.markings}
\tikzstyle arrowstyle=[scale=1]
\tikzstyle directed=[postaction={decorate,decoration={markings,
    mark=at position .65 with {\arrow[arrowstyle]{stealth}}}}]
\tikzstyle reverse directed=[postaction={decorate,decoration={markings,
    mark=at position .65 with {\arrowreversed[arrowstyle]{stealth};}}}]

\begin{document}

\begin{tikzpicture}[>=stealth]

 % Coordinates of origin and point of entry

\def\rwater{1.333};
\def\rglass{1.5};

\def\incidentangle{50};

  % Calculation of the angle of refraction
  %air-water
  %\pgfmathsetmacro{\refractedangle}{asin(sin(\incidentangle)/\rwater)};
  %air-glass
  \pgfmathsetmacro{\refractedangle}{asin(sin(\incidentangle)/\rglass)};
  
\clip (-4,-4) rectangle (4,4);

    % define coordinates
    \coordinate (TL) at (-4, 4);
    \coordinate (TR) at (4,4);
    \coordinate (BR) at (4,-4);
    \coordinate (BL) at (-4,-4);
 
    \coordinate (O) at (0,0) ;
    \coordinate (A) at (0,4) ;
    \coordinate (B) at (0,-4) ;
    
    % media
    \fill[blue!25!,opacity=.3] (-4,0) rectangle (4,4);
    \fill[blue!60!,opacity=.3] (-4,0) rectangle (4,-4);
    \node[right] at (-3.5,.25) {Air};
    \node[right] at (-3.5,-.25) {Glass};
    %\node[left] at (-2,-2) {Water};

    % axis
    \draw[dash pattern=on5pt off3pt] (A) -- (B);

    % rays
    \path [name path=edge] (TL) -- (TR) -- (BR) -- (BL) -- cycle;
    \path [name path=incident ray] (90+\incidentangle:10) -- (0,0);
    \path [name path=reflected ray] (0,0) -- (90-\incidentangle:10);
    \path [name path=refracted ray] (0,0)--(270+\refractedangle:10);  
   \draw [name intersections={of=incident ray and edge, by=t}] [red,ultra thick,reverse directed] (0,0)--
   node[near end,sloped,above]{incident ray} (t);
    \draw [name intersections={of=reflected ray and edge, by=t}] [red!20,ultra thick,directed] (0,0)--
   node[near end,sloped,above]{reflected ray} (t);
  \draw [name intersections={of=refracted ray and edge, by=t}] [blue!80,ultra thick,directed] (0,0)--
  node[near end,sloped,above]{refracted ray} (t);

    % angles
    \draw [<->](0,1) arc (90:90+\incidentangle:1);
    \draw [<->](0,1) arc (90:90-\incidentangle:1);
    \draw [<->](0,-1.4) arc (270:270+\refractedangle:1.4);
    \node[] at (270+\refractedangle/2:1.8)  {$\theta_{2}$};
    \node[] at (90+\incidentangle/2:1.4)  {$\theta_{1}$};
    \node[] at (90-\incidentangle/2:1.4)  {$\theta_{1}$};
    
\end{tikzpicture}
\end{document}