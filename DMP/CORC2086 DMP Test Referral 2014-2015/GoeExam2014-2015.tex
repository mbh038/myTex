% ANUfinalexam.tex (Version 2.0)
% ===============================================================================
% Australian National University Final Exam LaTeX template.
% 2004; 2009, Timothy Kam, ANU School of Economics
% Licence type: Free as defined in the GNU General Public Licence: http://www.gnu.org/licenses/gpl.html

\documentclass[a4paper,12pt,fleqn]{article}
\setlength{\parindent}{0em}
\usepackage{amsmath}
\usepackage{fancyhdr}
\usepackage{siunitx}
\usepackage{enumitem}
\usepackage{amsmath}
\usepackage{graphicx}
\usepackage{tikz}
\usepackage{import}
\usepackage{comment}

% Unit definitions %%%%%%%%%%%%%%%%%%%%%%%%%%%%%%%%%%

\DeclareSIUnit\kilowatthour{kWh}
\DeclareSIUnit\kilowattpeak{kW_P}
\DeclareSIUnit\kVA{kVA}
\DeclareSIUnit\kVAR{kVAR}
\DeclareSIUnit\year{y}
\DeclareSIUnit\north{N}
\DeclareSIUnit\south{S}
\DeclareSIUnit\second{s}


% Insert your course information here %%%%%%%%%%%%%%%%%%%%%%%%%%%%%%%%%%

\newcommand{\institution}{CORNWALL COLLEGE}
\newcommand{\titlehd}{BSc Renewable Energy and Carbon Management}
\newcommand{\examtype}{Referral Test}
\newcommand{\examdate}{Academic Year 2014-2015}
\newcommand{\examcode}{CORC2086}
\newcommand{\examtitle}{Data Modelling and Processing}
\newcommand{\readtime}{15 Minutes}
\newcommand{\writetime}{Two Hours}
\newcommand{\materials}{Non-programmable Calculators; Formula Sheet}
\newcommand{\middlewords}{Test continues on next page}
\newcommand{\lastwords}{End of Test}

%%%%%%%%%%%%%%%%%%%%%%%%%%%%%%%%%%%%%%%%%%%%%%%%%%%%

%\setcounter{MaxMatrixCols}{10}
\newtheorem{theorem}{Theorem}
\newtheorem{acknowledgement}[theorem]{Acknowledgement}
\newtheorem{algorithm}[theorem]{Algorithm}
\newtheorem{axiom}[theorem]{Axiom}
\newtheorem{case}[theorem]{Case}
\newtheorem{claim}[theorem]{Claim}
\newtheorem{conclusion}[theorem]{Conclusion}
\newtheorem{condition}[theorem]{Condition}
\newtheorem{conjecture}[theorem]{Conjecture}
\newtheorem{corollary}[theorem]{Corollary}
\newtheorem{criterion}[theorem]{Criterion}
\newtheorem{definition}[theorem]{Definition}
\newtheorem{example}[theorem]{Example}
\newtheorem{exercise}[theorem]{Exercise}
\newtheorem{lemma}[theorem]{Lemma}
\newtheorem{notation}[theorem]{Notation}
\newtheorem{problem}[theorem]{Problem}
\newtheorem{proposition}[theorem]{Proposition}
\newtheorem{remark}[theorem]{Remark}
\newtheorem{solution}[theorem]{Solution}
\newtheorem{summary}[theorem]{Summary}
\newenvironment{proof}[1][Proof]{\noindent\textbf{#1.} }{\ \rule{0.5em}{0.5em}}

% ANU Exams Office mandated margins and footer style
\setlength{\topmargin}{0cm}
\setlength{\textheight}{9.25in}
\setlength{\oddsidemargin}{0.0in}
\setlength{\evensidemargin}{0.0in}
\setlength{\textwidth}{16cm}
\pagestyle{fancy}
\lhead{} 
\chead{} 
\rhead{} 
\lfoot{} 
\cfoot{\footnotesize{Page \thepage \ of \pageref{finalpage} -- \titlehd \ (\examcode)}} 
\rfoot{} 

% DEPRECATED: ANU Exams Office mandated margins and footer style
%\setlength{\topmargin}{0cm}
%\setlength{\textheight}{9.25in}
%\setlength{\oddsidemargin}{0.0in}
%\setlength{\evensidemargin}{0.0in}
%\setlength{\textwidth}{16cm}
%\pagestyle{fancy}
%\lhead{} %left of the header
%\chead{} %center of the header
%\rhead{} %right of the header
%\lfoot{} %left of the footer
%\cfoot{} %center of the footer
%\rfoot{Page \ \thepage \ of \ \pageref{finalpage} \\
%       \texttt{\examcode}} %Print the page number in the right footer

\renewcommand{\headrulewidth}{0pt} %Do not print a rule below the header
\renewcommand{\footrulewidth}{0pt}


\begin{document}

% Title page

\begin{center}
%\vspace{5cm}
\large\textbf{\institution}
\end{center}
\vspace{1cm}

\begin{center}
\textit{ \examtype -- \examdate}
\end{center}
\vspace{1cm}

\begin{center}
\large\textbf{\titlehd}
\end{center}

\begin{center}
\large\textbf{\examcode}
\end{center}
\begin{center}
\large\textbf{\examtitle}
\end{center}
\vspace{4cm}
\vspace{4cm}

\begin{center}
%\textit{Reading Time: \readtime}
\end{center}
\begin{center}
\textit{Time Allowed:  \writetime}
\end{center}
\begin{center}
\textit{Permitted Materials: \materials}
\end{center}

% End title page
\newpage
\textbf{Formulae}
\newline
%\begin{comment}
\begin{itemize}

\item If $y=f(g(x))$, then if $u=g(x)$,\[\dfrac{dy}{dx}=\dfrac{dy}{du}\cdot\dfrac{du}{dx}\] .
\newline
\item If $y =uv$, where $u$ and $v$ are both functions of $x$, \[\dfrac{dy}{dx}=v\dfrac{du}{dx}+u\dfrac{dv}{dx}\] 
\newline
\item If $y=\dfrac{u}{v}$, where $u$ and $v$ are both functions of $x$, \[\dfrac{dy}{dx}=\dfrac{v\dfrac{du}{dx}-u\dfrac{dv}{dx}}{v^2}\]
\item \[\int u\dfrac{dv}{dx}dx=uv-\int v\dfrac{du}{dx}dx\]
\end{itemize}

%\end{comment}

\newpage

\begin{quote}
\begin{center}
\textit{There are\textbf{\ two} questions.}
\end{center}
\end{quote}

\begin{quote}
\begin{center}
\textit{Answer\textbf{\ all} parts of \textbf{\ both} questions.}
\end{center}
\end{quote}

\begin{quote}
\begin{center}
\textit{Each question is worth 50\% of the total marks}
\end{center}
\end{quote}

\begin{quote}
\begin{center}
\textit{Answers are expected to contain clear mathematical workings for all steps.}
\end{center}
\end{quote}

\bigskip

\newpage
\paragraph{\textbf{Question 1}}

Differentiate the following with respect to $x$

\begin{enumerate}[label=\alph*)]
\item $y=\left(x^3+2x^{-2}\right)^7$
\newline
\item $y=\sin\left(x^3+3x\right)$
\newline
\item $y=\ln\left(3x^2+1\right)$
\newline
\item $y=x^2e^{2x}$
\newline
\item $y=x^{\frac{1}{2}}\cos\left(2x\right)$
\newline
\item $y=\dfrac{x^2-\sin x}{x^3-1}$
\newline
\item $y=\dfrac{4x\,e^{x^2}}{\cos x}$
\end{enumerate}

\begin{center}
\vspace{3cm}
--------- \textit{\middlewords} ---------
\end{center}

\newpage
\paragraph{\textbf{Question 2}}

\begin{enumerate}[label=\alph*)]
\item Find \[\int^{\frac{1}{2}}_{-1}8x^2\left(x+4\right)\,dx\]
\item Find \[\int\left(4x+3\right)^6\,dx\]
\item Find \[\int^2_1 2x\,e^{3x}\,dx\]
\item Find \[\int x\sin3x\,dx\]
\item Using the substitution $u=3x^2-1$ or otherwise, find \[\int^2_0 x\left(3x^2-1\right)^2\,dx\]
\item Using the substitution $u=x^3+1$ or otherwise, find the exact value of \[\int^2_1 \dfrac{x^2}{x^3+1}\,dx\]
\end{enumerate}

\begin{center}
\vspace{3cm}
--------- \textit{\lastwords} ---------
\end{center}


\label{finalpage}

\begin{comment} % solutions

\newpage
\paragraph{\textbf{Solutions} \ }

\paragraph{\textbf{Solutions to Question 1: }}
\begin{enumerate}[label=\alph*)]
\item sdfsdf
\end{enumerate}

\paragraph{\textbf{Solutions to Question 2: }}
\begin{enumerate}[label=\alph*)]
\item sdfsdf
\end{enumerate}


\end{comment}

\end{document}

